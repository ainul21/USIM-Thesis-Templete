% Author Declaration Section
\addcontentsline{toc}{chapter}{AUTHOR DECLARATION}

\setlength{\parindent}{0pt}  % Disable indentation in front matter

\noindent
\begin{center}
    \vspace*{-10mm} % Top margin
    {\fontsize{12pt}{14pt}\selectfont \textbf{AUTHOR DECLARATION}}
\end{center}

\vspace{10mm} % Space between title and paragraph

\noindent
I hereby declare that the work in this thesis is my own except for quotations and summaries which have been duly acknowledged.

\vspace{35mm} % Space between text and the table

% Invisible table with 4 rows and 4 columns to align the date, name, signature, etc.
\noindent
\begin{tabular}{p{1cm}p{4cm}p{2cm}p{6cm}}
    \textbf{Date:} & {\fulldate} & \textbf{Signature:} & \\
     & & \textbf{Name:} & {\authorname} \\
     & & \textbf{Matric No:} & {\matrixnumber}\\
     & & \textbf{Address:} & \parbox[t]{6cm}{\address} \\
\end{tabular}

\vspace{20mm} % Space between table and footer

\clearpage

% Acknowledgements
\addcontentsline{toc}{chapter}{ACKNOWLEDGEMENTS}

\noindent
\begin{center}
    \vspace*{-5mm} % Top margin
    {\fontsize{12pt}{14pt}\selectfont \textbf{ACKNOWLEDGEMENTS}}
\end{center}

\vspace{10mm} % Space between title and paragraph

\noindent
\textit{Bismillahirrahmanirrahim}.

All praises and gratitude are due to Allah SWT for granting me the strength, patience, and perseverance to complete this research project. I express my deepest gratitude to the Messenger Muhammad SAW, whose teachings continue to be a source of guidance and inspiration in my life. \\

I would like to express my deepest gratitude to those who have supported me throughout the course of this research. First and foremost, I would like to thank my supervisor, [Supervisor's Name], for their invaluable guidance, encouragement, and insight during the development of this thesis. Their expertise and unwavering support have been instrumental in shaping this work.\\

I am also deeply grateful to the faculty and staff at [Your University] for providing me with the resources and knowledge necessary to pursue my research. Special thanks go to my committee members, [Committee Member 1] and [Committee Member 2], for their constructive feedback and thoughtful suggestions, which greatly enhanced the quality of this thesis.\\

My heartfelt thanks go to my family and friends for their constant encouragement and understanding throughout this journey. Your unwavering belief in me has been a source of motivation during challenging times.\\

Finally, I would like to acknowledge my colleagues and peers at [Your University or Research Group] for their support, collaboration, and the stimulating discussions that have enriched my academic experience. Your camaraderie has made this journey a rewarding one.\\

Thank you all for your support and belief in my work.\\

\textit{Alhamdulillah}, I am grateful for this journey and for all those who have walked with me along the way.



\clearpage

% Abstract in Malay
\addcontentsline{toc}{chapter}{ABSTRAK}

\noindent
\begin{center}
    \vspace*{0mm} % Top margin
    {\fontsize{12pt}{14pt}\selectfont \textbf{ABSTRAK}}
\end{center}

\vspace{10mm} % Space between title and paragraph

\noindent
Tesis ini membincangkan keberkesanan algoritma hipotesis dalam meningkatkan prestasi sistem yang disimulasikan. Objektif utama kajian ini adalah untuk menilai kecekapan algoritma yang dicadangkan dalam pelbagai senario. Hasil kajian menunjukkan bahawa kaedah yang dicadangkan mampu memberikan prestasi yang lebih baik berbanding pendekatan tradisional, sekaligus menunjukkan potensi algoritma tersebut untuk aplikasi masa depan.

\clearpage

% Abstract in English
\addcontentsline{toc}{chapter}{ABSTRACT}

\noindent
\begin{center}
    \vspace*{0mm} % Top margin
    {\fontsize{12pt}{14pt}\selectfont \textbf{ABSTRACT}}
\end{center}

\vspace{10mm} % Space between title and paragraph

\noindent
This thesis explores the impact of hypothetical algorithms on the performance of simulated systems. The primary objective is to evaluate the efficiency of these algorithms in various scenarios. Results indicate that the proposed methods outperform traditional approaches, demonstrating their potential for future applications.

\clearpage


% Abstract in Arabic
\addcontentsline{toc}{chapter}{AL-MULAKHKHAS}

\noindent
\begin{center}
    \vspace*{0mm} % Top margin
    {\fontsize{12pt}{14pt}\selectfont \textbf{الملخص}}
\end{center}

\vspace{10mm} % Space between title and paragraph

\noindent
يلعب قطاع نخيل الزيت دورًا حيويًا في سوق الوقود الحيوي العالمي، حيث يوفر مصدرًا مستدامًا للمواد الخام لإنتاج الوقود الحيوي. تقدم هذه الدراسة نهجًا متكاملاً لاكتشاف العناقيد الطازجة لثمار نخيل الزيت (FFB) باستخدام تقنيات متقدمة في معالجة الصور والتعلم العميق. من خلال الاستفادة من مجموعة من طرق المعالجة المسبقة المبتكرة، والتصنيف المعتمد على المنطق الضبابي، والنماذج المتقدمة للتعلم العميق، يهدف نهجنا إلى تعزيز دقة وكفاءة اكتشاف FFB في ظروف بيئية متنوعة. يتم استخدام شبكة استعادة الصور متعددة المقاييس (MIRNet) لتحسين جودة الصور في ظروف الإضاءة المنخفضة، بينما يتم تعديل نموذج YOLOv8 المخصص، الذي يُطلق عليه اسم YOLOv8-FFB، لتحقيق اكتشاف دقيق لعناقيد الثمار. يُظهر هذا النظام المتكامل أداءً متميزًا في اكتشاف وتحليل FFB، مما يوفر تقدمًا كبيرًا في تقدير المحاصيل الآلي وإدارة المزارع. تشير النتائج إلى أن طريقتنا قادرة على التغلب على التحديات التي تفرضها اختلافات الإضاءة، مما يسهم في تحسين عمليات إنتاج الوقود الحيوي واستدامة قطاع نخيل الزيت.

\clearpage